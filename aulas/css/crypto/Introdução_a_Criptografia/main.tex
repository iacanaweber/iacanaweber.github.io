\documentclass{beamer}

\usetheme{Madrid}
\usecolortheme{beaver}
\usepackage[utf8]{inputenc}
\usepackage[brazil]{babel}
\usepackage{graphicx}
\usepackage{xcolor}
\usepackage{tikz}
\usetikzlibrary{shapes, arrows.meta, positioning}
\usepackage{array} 

\usepackage{tcolorbox}     % Para caixas coloridas e formatadas
\tcbuselibrary{skins, breakable}  % Habilita opções avançadas do tcolorbox

\usepackage{graphicx}      % Necessário para \scalebox

\AtBeginSection[]
{
  \begin{frame}
    \frametitle{Índice}
    \tableofcontents[currentsection]
  \end{frame}
}

% Informações da apresentação
\title[Introdução à Criptografia]{Introdução à Criptografia}
\author{Prof. Iaçanã Ianiski Weber}
\institute{PUCRS}
\date{CSS (98G08-4)}

\begin{document}

% Slide de Capa Personalizado
\begin{frame}[plain]

% Linha superior com logo e título
\noindent
\begin{minipage}[t]{0.15\textwidth}
    \includegraphics[width=\linewidth]{img/Brasão_PUCRS.png}
\end{minipage}%
\begin{minipage}[t]{0.75\textwidth}
    \begin{center}
        \raisebox{5ex}{\LARGE\bfseries\textcolor{blue}{Introdução à Criptografia}}
    \end{center}
\end{minipage}

\vspace{2cm}

\begin{center}
    \Large \textbf{Prof. Dr. Iaçanã Ianiski Weber} \\
    \vspace{0.2cm}
    \normalsize \textit{Confiabilidade e Segurança de Software} \\
    \vspace{0.2cm}
    \large 98G08-4
\end{center}

\vfill

\begin{center}
    \tiny \textit{Agradecimentos especiais ao Prof. Avelino Zorzo e aos Autores Christof Paar e Jan Pelzl pelo material.}
\end{center}

\end{frame}

% Slide de Índice
\begin{frame}{Índice}
    \tableofcontents
\end{frame}

%%%%%%%%%%%%%%%%%%%%%%%%%%%%%%%%%%%
%%%%%%%%%%%%%%%%%%%%%%%%%%%%%%%%%%%
%%%%%%%%%%%%%%%%%%%%%%%%%%%%%%%%%%%

% Seções principais
\section{Visão geral da Criptologia}

\begin{frame}{Objetivos Clássicos da Segurança da Informação}

\begin{enumerate}
    \item \textbf{\textcolor{blue}{P}rivacidade}: sem vazar dados confidenciais
    \item \textbf{\textcolor{blue}{A}utenticação}: sem se passar por outro
    \item \textbf{\textcolor{blue}{I}ntegridade}: sem alteração
    \item \textbf{\textcolor{blue}{N}ão-repúdio}: não ser capaz de negar
\end{enumerate}

\vspace{1em}

\textit{Esses quatro pilares formam a base da \textbf{segurança} em sistemas computacionais.}

\end{frame}


\begin{frame}{Leituras e Informações Adicionais}

\textit{Paar, Christof, and Jan Pelzl}. \textbf{Understanding Cryptography: A Textbook for Students and Practitioners}. Springer, 2010.


\vspace{0.5cm}
\textbf{Complementar ao \textit{Understanding Cryptography}:}
\begin{itemize}
    \item A. Menezes, P. van Oorschot, S. Vanstone, \textit{Handbook of Applied Cryptography}. CRC Press, outubro de 1996.
    \item H. v. Tilborg (ed.), \textit{Encyclopedia of Cryptography and Security}, Springer, 2005.
\end{itemize}

\vspace{0.5cm}
\textbf{História da Criptografia (ótima leitura para a noite):}
\begin{itemize}
    \item S. Singh, \textit{The Code Book: The Science of Secrecy from Ancient Egypt to Quantum Cryptography}, Anchor, 2000.
    \item D. Kahn, \textit{The Codebreakers: The Comprehensive History of Secret Communication from Ancient Times to the Internet}. 2ª edição, Scribner, 1996.
\end{itemize}

\end{frame}

\begin{frame}{Classificação da Área da Criptologia}
\centering
\scalebox{0.75}{ % adjust the scale factor here (try 0.65–0.85 as needed)
\begin{tikzpicture}[
  every node/.style={draw, ellipse, minimum width=2.5cm, minimum height=0.8cm, align=center, font=\footnotesize},
  ->, >=Stealth, thick,
  node distance=0.8cm and 0.8cm
  ]

% Definição dos nós
\node (root) {Criptologia};
\node (crypto) [below left=of root] {Criptografia};
\node (analysis) [below right=of root] {Criptoanálise};

\node (sym) [below left=1.4cm and 1.2cm of crypto] {Cifras Simétricas};
\node (asym) [below=1.8cm of crypto] {Cifras Assimétricas};
\node (proto) [below right=1.4cm and 1.2cm of crypto] {Protocolos};

\node (block) [below left=1.4cm and 0.8cm of sym] {Cifras de Bloco};
\node (stream) [below right=1.4cm and 0.8cm of sym] {Cifras de Fluxo};

% Conexões
\draw (root) -- (crypto);
\draw (root) -- (analysis);

\draw (crypto) -- (sym);
\draw (crypto) -- (asym);
\draw (crypto) -- (proto);

\draw (sym) -- (block);
\draw (sym) -- (stream);

\end{tikzpicture}
}
\end{frame}


\begin{frame}{Alguns Fatos Básicos}
\begin{itemize}
    \item \textbf{Criptografia Antiga:} Há indícios de criptografia no Egito por volta de 2000 a.C.\\
    Esquemas de substituição por letras (ex.: cifra de César) populares desde então.
    
    \item \textbf{Cifras Simétricas:} Todos os esquemas criptográficos desde a antiguidade até 1976 eram simétricos.

    \item \textbf{Cifras Assimétricas:} Em 1976, a criptografia de chave pública (ou assimétrica) foi proposta abertamente por Diffie, Hellman e Merkle.

    \item \textbf{Esquemas Híbridos:} A maioria dos protocolos atuais são esquemas híbridos, ou seja, utilizam ambos:
    \begin{itemize}
        \item Cifras simétricas (ex.: para encriptação e autenticação de mensagens)
        \item Cifras assimétricas (ex.: para troca de chaves e assinatura digital)
    \end{itemize}
\end{itemize}
\end{frame}



%%%%%%%%%%%%%%%%%%%%%%%%%%%%%%%%%%%
%%%%%%%%%%%%%%%%%%%%%%%%%%%%%%%%%%%
%%%%%%%%%%%%%%%%%%%%%%%%%%%%%%%%%%%
\section{Conceitos Básicos de Criptografia Simétrica}
\begin{frame}{Criptografia Simétrica}
\begin{itemize}
    \item Nomes alternativos: \textbf{private-key}, \textbf{single-key} ou \textbf{secret-key} cryptography.
\end{itemize}

\begin{center}
    \includegraphics[width=0.75\linewidth]{img/symm_crypto.png}
\end{center}

\begin{itemize}
    \item \textbf{Problema:}
    \begin{enumerate}
        \item Alice e Bob desejam se comunicar por um canal inseguro (ex.: WLAN ou Internet).
        \item Um terceiro malicioso, Oscar (o "cara mau"), tem acesso ao canal, mas não deve ser capaz de entender a comunicação.
    \end{enumerate}
\end{itemize}
\end{frame}


\begin{frame}{Criptografia Simétrica}
\scalebox{1.0}{
\parbox{\linewidth}{
\textbf{Solução:} Criptografia com cifra simétrica.\\
\hspace{1em} $\Rightarrow$ Oscar obtém apenas o \textit{ciphertext} $y$, que se parece com uma sequência aleatória de bits.

\vspace{-0.5em}
\begin{center}
    \includegraphics[width=0.8\linewidth]{img/symm_crypto_solution.png}
\end{center}
\vspace{-0.8em}

\begin{itemize}
    \item $x$ é o \textbf{plaintext}
    \item $y$ é o \textbf{ciphertext}
    \item $K$ é a \textbf{key}
    \item O conjunto de todas as chaves $\{K_1, K_2, \ldots, K_n\}$ é o \textbf{key space}
\end{itemize}
}}
\end{frame}


\begin{frame}{Criptografia Simétrica}
\vspace{-0.5em}
\begin{block}{Equações}
\begin{itemize}
    \item \textbf{Equação de encriptação:} \quad $y = e_K(x)$
    \item \textbf{Equação de desencriptação:} \quad $x = d_K(y)$
\end{itemize}
\end{block}

\vspace{0.5em}
\begin{itemize}
    \item Encriptação e desencriptação são operações inversas se a mesma chave $K$ for usada em ambos os lados: \\
    \begin{center}
        $ d_K(y) = d_K(e_K(x)) = x $    
    \end{center}
    
    \item \textbf{Importante:} a chave deve ser transmitida por um \textbf{canal seguro} entre Alice e Bob.

    \item Esse canal seguro pode ser realizado, por exemplo, com a instalação manual da chave (como no protocolo Wi-Fi WPA) ou via um mensageiro de confiança.

    \item O sistema só é seguro se um atacante não obtiver conhecimento da chave $K$!

    \item[\textcolor{orange}{$\Rightarrow$}] \textcolor{orange}{O problema da comunicação segura se reduz à transmissão e armazenamento seguro da chave $K$.}
\end{itemize}
\end{frame}



%%%%%%%%%%%%%%%%%%%%%%%%%%%%%%%%%%%
%%%%%%%%%%%%%%%%%%%%%%%%%%%%%%%%%%%
%%%%%%%%%%%%%%%%%%%%%%%%%%%%%%%%%%%
\section{Criptoanálise}
\begin{frame}{Por que precisamos da Criptoanálise?}
\vspace{-0.5em}
\begin{itemize}
    \item Não existe nenhuma \textit{mathematical proof of security} para qualquer cifra prática.
    \item A única forma de garantir que uma cifra é segura é tentar quebrá-la (e falhar)!
\end{itemize}

%\vspace{0.5em}
\textbf{O Princípio de Kerckhoff} é fundamental na criptografia moderna:

\begin{block}{}
\centering
\textit{Um sistema criptográfico deve ser seguro mesmo que o atacante (Oscar) conheça todos os detalhes sobre o sistema, com exceção da chave secreta.}
\end{block}

%\vspace{0.5em}
\begin{itemize}
    \item Para atingir o Princípio de Kerckhoff na prática: \\
    \textbf{Utilize apenas cifras amplamente conhecidas que tenham sido criptoanalisadas por anos por bons criptógrafos!} \\

    \item \textbf{Observação:} Pode parecer que uma cifra é “mais segura” se seus detalhes forem mantidos secretos. Porém, a história mostra que cifras secretas quase sempre foram quebradas uma vez que foram revertidas por engenharia reversa. \\
    (Exemplo: CSS – Content Scrambling System – usado para proteção de conteúdo em DVDs.)
\end{itemize}
\end{frame}


\begin{frame}{Criptoanálise: Atacando Sistemas Criptográficos}
\begin{center}
    \includegraphics[width=0.8\linewidth]{img/cryptanalysis.png}
\end{center}

\vspace{0.5em}
\begin{itemize}
    \item \textbf{Ataques Clássicos}
    \begin{itemize}
        \item Análise Matemática
        \item Ataque de Força Bruta
    \end{itemize}

    \item \textbf{Ataques de Implementação:} Tentam extrair a chave por engenharia reversa ou medição de consumo de energia, por exemplo, em cartões inteligentes bancários.

    \item \textbf{Engenharia Social:} Por exemplo, enganar um usuário para que ele revele sua senha.
\end{itemize}
\end{frame}



\begin{frame}{Brute-Force Attack contra Cifras Simétricas}
\vspace{-0.3em}
\begin{itemize}
    \item Trata a cifra como uma \textit{black box}.
    \item Requer (ao menos) um par \textit{plaintext-ciphertext} $(x_0, y_0)$.
    \item Verifica todas as chaves possíveis até que a condição seja satisfeita:
    \begin{center}
        $d_K(y_0) \stackrel{?}{=} x_0$
    \end{center}
    \item Quantas chaves precisamos considerar?
\end{itemize}
\vspace{0.5em}
\small
\resizebox{\linewidth}{!}{%
\begin{tabular}{|c|c|>{\centering\arraybackslash}m{7cm}|}
\hline
\textbf{Key length (bits)} & \textbf{Key space} & \textbf{Security life time} \\
\hline
64  & $2^{64}$  & \textbf{Obsoleto / não aprovado pelo NIST} (força menor que 112 bits) \\
\hline
128 & $2^{128}$ & \textbf{Longo prazo (clássico)}; planejar migração para \textbf{PQC}; contra quântico oferece $\sim\!64$ bits (insuficiente) \\
\hline
256 & $2^{256}$ & \textbf{Longo prazo com ampla margem (clássico)}; contra quântico $\sim\!128$ bits (mitiga Grover); recomendado para alvos de muito alto valor e/ou alinhamento a CNSA 2.0 \\
\hline
\end{tabular}%
}


\vspace{-0.1em}
\smallskip
\textbf{Importante:} Um atacante precisa ter sucesso em apenas \textbf{um} ataque. Logo, um grande espaço de chaves não ajuda se outros ataques (como \textit{engenharia social}) forem possíveis.
\end{frame}





%%%%%%%%%%%%%%%%%%%%%%%%%%%%%%%%%%%
%%%%%%%%%%%%%%%%%%%%%%%%%%%%%%%%%%%
%%%%%%%%%%%%%%%%%%%%%%%%%%%%%%%%%%%
\section{Cifras de Substituição}
\begin{frame}{Cifras de Substituição}
\begin{itemize}
    \item Cifra histórica
    \item Excelente ferramenta para entender brute-force vs. ataques analíticos
    \item Cifra letras em vez de bits (como todas as cifras até o fim da Segunda Guerra Mundial)
\end{itemize}

\textbf{Ideia:} substituir cada letra do \textit{plaintext} por outra letra fixa.

\begin{center}
\begin{tabular}{c c c}
\textbf{Plaintext} & $\rightarrow$ & \textbf{Ciphertext} \\
A & $\rightarrow$ & k \\
B & $\rightarrow$ & d \\
C & $\rightarrow$ & w \\
\multicolumn{3}{c}{\dots} \\
\end{tabular}
\end{center}

Por exemplo, \texttt{ABBA} seria encriptado como \texttt{kddk}.

\begin{itemize}
    \item \textbf{Exemplo (ciphertext):} \\
    \begin{center}
        \texttt{iq ifcc vqqr fb rdq vfllcq na rdq cfjwhwz hr bnnb hcc hwwhbsqvqbre hwq vhlq}
    \end{center}
    
    \item Quão segura é a cifra de substituição? Vamos analisar os ataques...
\end{itemize}
\end{frame}


\begin{frame}{Ataques contra a Cifra de Substituição}
\begin{enumerate}
    \item \textbf{Ataque: Busca Exaustiva por Chave (Brute-Force Attack)}
\end{enumerate}

\begin{itemize}
    \item Basta tentar todas as tabelas de substituição possíveis até que um \textit{plaintext} inteligível apareça.\\
    \small 
    (Note que cada tabela de substituição é uma chave.)

    \normalsize	
    \item Quantas tabelas de substituição existem (= chaves)?
    \begin{center}
        $26 \times 25 \times \dots \times 3 \times 2 \times 1 = 26! \approx 2^{88}$
    \end{center}
    
    \item \textbf{Buscar em $2^{88}$ chaves é inviável com os computadores atuais!} \\
    
    \item \textbf{Q:} Podemos então concluir que a \textit{substitution cipher} é segura já que um ataque de força bruta não é viável?

    \item \textbf{A:} Não! Precisamos nos proteger contra \textbf{todos} os tipos de ataques…
\end{itemize}
\end{frame}


\begin{frame}{Análise Matemática/Estatística}
\begin{enumerate}
    \setcounter{enumi}{1}
    \item \textbf{Ataque: Análise de Frequência de Letras}
\end{enumerate}
\begin{itemize}
    \item As letras têm frequências muito diferentes na maioria das línguas.
    \item Além disso, a frequência das letras do \textit{plaintext} é preservada no \textit{ciphertext}.
    \item Por exemplo, \textit{e} é a letra mais comum em inglês — quase 13\% de todas as letras em um texto típico em inglês são \textit{e}.
    \item A próxima letra mais comum é \textit{t}, com cerca de 9\%.
\end{itemize}

\begin{center}
    \includegraphics[width=0.7\linewidth]{img/letters_english.png}
\end{center}
\end{frame}


\begin{frame}{Aplicando a Análise de Frequência de Letras}
\begin{itemize}
    \item Vamos retornar ao nosso exemplo e identificar a letra mais frequente:\\
    \begin{center}
        {\ttfamily
        i{\color{orange}q} ifcc v{\color{orange}q}{\color{orange}q}r fb rd{\color{orange}q} vfllc{\color{orange}q} na rd{\color{orange}q} cfjwhwz hr bnnb hcc hwwhbs{\color{orange}q}v{\color{orange}q}bre hw{\color{orange}q} vhl{\color{orange}q}
        }    
    \end{center}
    \vspace{1em}
    \item Substituímos a letra \texttt{q} do \textit{ciphertext} por \texttt{E} e obtemos:\\
    \begin{center}
        {\ttfamily
        i{\color{orange}E} ifcc v{\color{orange}E}{\color{orange}E}r fb rd{\color{orange}E} vfllc{\color{orange}E} na rd{\color{orange}E} cfjwhwz hr bnnb hcc hwwhbs{\color{orange}E}v{\color{orange}E}bre hw{\color{orange}E} vhl{\color{orange}E}
        }    
    \end{center}
    \vspace{1em}
    \item Com mais substituições baseadas na frequência das letras restantes, obtemos o \textit{plaintext}:
    \begin{center}
        {\color{orange}\ttfamily
        WE WILL MEET IN THE MIDDLE OF THE LIBRARY AT NOON 
        ALL ARRANGEMENTS ARE MADE}    
    \end{center}
    
\end{itemize}
\end{frame}


\begin{frame}{Aplicando a Análise de Frequência de Letras}
\begin{itemize}
    \item Na prática, não apenas as frequências de letras individuais podem ser usadas para ataques, mas também a frequência de pares de letras (ex.: \textit{th} é muito comum em inglês), trios de letras, etc.

\end{itemize}

\vspace{1em}
\textbf{Lição importante:} Mesmo que a \textit{\textbf{cifra de substituição }} tenha um espaço de chaves grande (aproximadamente $2^{88}$), ela pode ser facilmente quebrada por métodos analíticos. 

Este é um excelente exemplo de que um esquema de encriptação precisa ser resistente a \textbf{todos os tipos de ataques}.
\end{frame}


%\begin{frame}{Exercício}
 %   Veja o Problema 1.1 do livro \textit{Understanding Cryptography} para um \textit{ciphertext} mais longo que você pode tentar quebrar!
%\end{frame}


%%%%%%%%%%%%%%%%%%%%%%%%%%%%%%%%%%%
%%%%%%%%%%%%%%%%%%%%%%%%%%%%%%%%%%%
%%%%%%%%%%%%%%%%%%%%%%%%%%%%%%%%%%%
\section{Aritmética Modular}
\begin{frame}{Breve Introdução à Aritmética Modular}
\textbf{Por que precisamos estudar aritmética modular?}
\begin{itemize}
    \item Extremamente importante para criptografia assimétrica (RSA, curvas elípticas, etc.)
    \item Algumas cifras históricas podem ser elegantemente descritas com aritmética modular (veremos a cifra de César mais adiante).
\end{itemize}
\end{frame}


\begin{frame}{Breve Introdução à Aritmética Modular}
De modo geral, a maioria dos sistemas criptográficos é baseada em \textbf{conjuntos de números} que são:
\begin{enumerate}
    \item \textbf{Discretos} (conjuntos com inteiros são particularmente úteis)
    \item \textbf{Finitos} (isto é, trabalhamos apenas com uma quantidade finita de números)
\end{enumerate}

Parece abstrato demais? --- Vamos observar um conjunto finito com números discretos com o qual estamos bem familiarizados: \textbf{um relógio}.

\vspace{-0.2em}
\begin{center}
    \includegraphics[width=0.25\linewidth]{img/relogio_modular.png}
\end{center}
\vspace{-0.5em}

Curiosamente, mesmo que os números aumentem a cada hora, nunca saímos do conjunto dos inteiros:
\begin{center}
    $1, 2, 3, \ldots, 11, 12, 1, 2, 3, \ldots$    
\end{center}

\end{frame}


\begin{frame}{Breve Introdução à Aritmética Modular}
\begin{itemize}
    \item Desenvolvemos agora um sistema aritmético que nos permite \textbf{calcular} em conjuntos finitos de inteiros, como os 12 números de um relógio.
    
    \item É fundamental ter uma operação que \textbf{mantém os números dentro de limites}, ou seja, após adição ou multiplicação, eles nunca devem sair do conjunto (ex.: nunca maiores que 12).
\end{itemize}

\begin{center}
\scalebox{0.6}{%
\begin{minipage}{\linewidth}
\begin{block}{Definição: Operação Módulo}
Sejam $a$, $r$, $m$ inteiros com $m > 0$. Escrevemos:
\[
a \equiv r \mod m
\]
se $(r - a)$ for divisível por $m$.
\begin{itemize}
    \item $m$ é chamado de \textbf{módulo}
    \item $r$ é chamado de \textbf{resto}
\end{itemize}
\end{block}
\end{minipage}
}
\end{center}

\vspace{0.5em}
\textbf{Exemplos de redução modular:}
\begin{itemize}
    \item Seja $a = 12$ e $m = 9$: \quad $12 \equiv 3 \mod 9$
    \item Seja $a = 37$ e $m = 9$: \quad $37 \equiv 1 \mod 9$
    \item Seja $a = -7$ e $m = 9$: \quad $-7 \equiv 2 \mod 9$
\end{itemize}

\smallskip
\textit{(Você deve verificar se a condição “$m$ divide $(r - a)$” vale nos três casos.)}
\end{frame}


\begin{frame}{Propriedades da Aritmética Modular (1)}
\begin{itemize}
    \item \textbf{O resto não é único}
\end{itemize}

É um pouco surpreendente, mas para um dado módulo $m$ e número $a$, existem (infinitamente) muitos restos válidos.

\vspace{0.5em}
\textbf{Exemplo:}
\begin{itemize}
    \item $12 \equiv 3 \mod 9$ $\rightarrow$ 3 é um resto válido pois $9 \mid (3 - 12)$

    \item $12 \equiv 21 \mod 9$ $\rightarrow$ 21 é um resto válido pois $9 \mid (21 - 12)$

    \item $12 \equiv -6 \mod 9$  $\rightarrow$ -6 é um resto válido pois $9 \mid (-6 - 12)$
\end{itemize}
\end{frame}



\begin{frame}{Propriedades da Aritmética Modular (2)}
\begin{itemize}
    \item \textbf{Qual resto devemos escolher?}
\end{itemize}

Por convenção, geralmente usamos o \textbf{menor número inteiro positivo} $r$ como o resto. Esse valor pode ser calculado como:
\[
a = q \cdot m + r \qquad \text{onde} \quad 0 \leq r \leq m - 1
\]

\textbf{Exemplo:} Seja $a = 12$ e $m = 9$:
\[
12 = 1 \times 9 + 3 \qquad \Rightarrow \quad r = 3
\]

\vspace{1em}
\textbf{Observação:} Essa escolha é apenas uma convenção. \\
Algoritmicamente, somos livres para usar qualquer outro resto válido ao calcular funções criptográficas.
\end{frame}



\begin{frame}{Propriedades da Aritmética Modular (3)}
\begin{itemize}
    \item \textbf{Como realizamos divisão modular?}
\end{itemize}

Em vez de realizar uma divisão direta, preferimos multiplicar pelo inverso modular. Por exemplo:
\begin{center}
    $b / a \equiv b \cdot a^{-1} \mod m$    
\end{center}

O inverso $a^{-1}$ de um número $a$ é definido como o número tal que:
\begin{center}
    $a \cdot a^{-1} \equiv 1 \mod m$    
\end{center}

\textbf{Exemplo:} Quanto é $5 / 7 \mod 9$?

O inverso de $7 \mod 9$ é $4$, pois $7 \cdot 4 = 28 \equiv 1 \mod 9$, portanto:
\begin{center}
    $5 / 7 \equiv 5 \cdot 4 = 20 \equiv 2 \mod 9$    
\end{center}

\end{frame}

\begin{frame}{Propriedades da Aritmética Modular (4)}
\begin{itemize}
    \item \textbf{Como se calcula o inverso?}
\end{itemize}
O inverso de um número $a \mod m$ existe se, e somente se:
\begin{center}
    $\gcd(a, m) = 1$    
\end{center}

(No exemplo acima, $\gcd(5, 9) = 1$, então o inverso de $5$ existe módulo $9$.)

\footnotesize
Atualmente, a melhor forma de calcular o inverso é por \textbf{busca exaustiva}. No futuro veremos o poderoso \textbf{Algoritmo de Euclides}, que permite encontrar o inverso modular dado um número e o módulo.
\normalsize

\end{frame}


\begin{frame}{Propriedades da Aritmética Modular (5)}
\begin{itemize}
    \item \textbf{A redução modular pode ser feita em qualquer etapa de um cálculo}
\end{itemize}

Vamos ver um exemplo. Queremos calcular $3^8 \mod 7$ (note que exponenciação é extremamente importante em criptografia de chave pública).

\vspace{0.5em}
\textbf{1. Abordagem: Exponenciação seguida da redução modular}

\[
3^8 = 6561 \equiv 2 \mod 7
\]

Note que o resultado intermediário foi 6561, embora o resultado final não possa ser maior que 6.

\end{frame}




\begin{frame}{Propriedades da Aritmética Modular (6)}
\textbf{2. Abordagem: Exponenciação com redução modular intermediária}

\[
3^8 = 3^4 \cdot 3^4 = 81 \cdot 81
\]

Neste ponto, reduzimos $81 \mod 7$:

\[
3^8 = 81 \cdot 81 \equiv 4 \cdot 4 \mod 7
\]
\[
4 \cdot 4 = 16 \equiv 2 \mod 7
\]


Note que podemos fazer todas essas multiplicações sem calculadora, enquanto calcular mentalmente $3^8 = 6561$ é bem desafiador para a maioria de nós.

\vspace{0.5em}
\begin{block}{Regra geral}
\centering
Para a maioria dos algoritmos, é vantajoso reduzir os resultados intermediários o mais cedo possível.
\end{block}
\end{frame}

%FALTAM 2 SLIDES AQUI!% 30 e 31!

%%%%%%%%%%%%%%%%%%%%%%%%%%%%%%%%%%%
%%%%%%%%%%%%%%%%%%%%%%%%%%%%%%%%%%%
%%%%%%%%%%%%%%%%%%%%%%%%%%%%%%%%%%%
\section{Cifra de César (ou Deslocamento)}
\begin{frame}{Cifra de Deslocamento (ou de César) (1)}
\begin{itemize}
    \item Cifra antiga, supostamente usada por Júlio César.
    \item Substitui cada letra do \textit{plaintext} por outra.
    \item Regra de substituição é muito simples: pega a letra que está $k$ posições adiante no alfabeto.
\end{itemize}

\vspace{0.5em}
É necessário mapear letras → números:

\begin{center}
\scriptsize
\begin{tabular}{|c|c|c|c|c|c|c|c|c|c|c|c|c|}
\hline
A & B & C & D & E & F & G & H & I & J & K & L & M \\
\hline
0 & 1 & 2 & 3 & 4 & 5 & 6 & 7 & 8 & 9 & 10 & 11 & 12 \\
\hline
\end{tabular}
\hspace{1em}
\begin{tabular}{|c|c|c|c|c|c|c|c|c|c|c|c|c|}
\hline
N & O & P & Q & R & S & T & U & V & W & X & Y & Z \\
\hline
13 & 14 & 15 & 16 & 17 & 18 & 19 & 20 & 21 & 22 & 23 & 24 & 25 \\
\hline
\end{tabular}
\end{center}

\vspace{1em}
\textbf{Exemplo para $k = 7$:}

\begin{itemize}
    \item \texttt{Plaintext} = \texttt{ATTACK} = 0, 19, 19, 0, 2, 10
    \item \texttt{Ciphertext} = \texttt{haaahr} = 7, 0, 0, 7, 9, 17
\end{itemize}

\vspace{0.5em}
Note que as letras \textit{“dão a volta”} no final do alfabeto, o que pode ser descrito matematicamente como redução módulo 26.\\
Exemplo: $19 + 7 = 26 \equiv 0 \mod 26$
\end{frame}

\begin{frame}{Cifra de Deslocamento (ou de César) (2)}
\begin{itemize}
    \item Descrição matemática elegante da cifra:
\end{itemize}

\vspace{0.5em}
\begin{block}{}
Sejam $k$, $x$, $y \in \{0, 1, \dots, 25\}$
\begin{itemize}
    \item \textbf{Encriptação:} \quad $y = e_k(x) \equiv x + k \mod 26$
    \item \textbf{Decriptação:} \quad $x = d_k(y) \equiv y - k \mod 26$
\end{itemize}
\end{block}

\vspace{0.5em}
\begin{itemize}
    \item \textbf{Pergunta:} A cifra de César é segura?
    \item \textbf{Resposta:} Não! Vários ataques são possíveis, incluindo:
    \begin{itemize}
        \item Busca exaustiva (espaço de chaves tem apenas 26 possibilidades!)
        \item Análise de frequência de letras (como na cifra de substituição)
    \end{itemize}
\end{itemize}
\end{frame}

\section{Cifra Afim (Affine)}
\begin{frame}[t]{Cifra Afim (Affine) (1)}
\small % slightly smaller text
\vspace{-1em}
% Top bullets (tighter spacing)
\begin{itemize}\setlength{\itemsep}{0.35em}
  \item Extensão da cifra de deslocamento: em vez de apenas somar a chave ao texto-claro, também multiplicamos por ela.
  \item Usamos uma chave com duas partes: $k=(a,b)$.
\end{itemize}
\vspace{-0.9em}
% Block without shadow to save space
{%
\setbeamertemplate{blocks}[rounded][shadow=false]
\begin{block}{}
Sejam $k, x, y \in \{0,1,\dots,25\}$.\\[-0.3em]
\begin{itemize}\setlength{\itemsep}{0.2em}
  \item \textbf{Encriptação:} \quad $y = e_k(x) \equiv a\,x + b \pmod{26}$
  \item \textbf{Decriptação:} \quad $x = d_k(y) \equiv a^{-1}(y - b) \pmod{26}$
\end{itemize}
\end{block}
}
\vspace{-0.25em}

% Bottom bullets (tighter spacing)
\begin{itemize}\setlength{\itemsep}{0.35em}
  \item Como o inverso de $a$ é necessário para decifrar, só podemos usar valores de $a$ tais que $\gcd(a,26)=1$.
  \item Existem 12 valores de $a$ que atendem a essa condição (por exemplo, $a \in \{1,3,5,7,9,11,15,17,19,21,23,25\}$).
  \item Disso decorre que o espaço de chaves é apenas $12 \times 26 = 312$.
  \item Novamente, vários ataques são possíveis, incluindo:
  \begin{itemize}\setlength{\itemsep}{0.2em}
    \item busca exaustiva e análise de frequência de letras, semelhante ao ataque contra a cifra de substituição.
  \end{itemize}
\end{itemize}
\end{frame}





\section{Conclusão}
\begin{frame}{Leis de Segurança de Shamir (Prêmio Turing 2002)}

\begin{itemize}
    \item \textbf{Sistemas completamente seguros não existem.}
    
    \item \textbf{Para diminuir suas vulnerabilidades pela metade,}
    
    \quad deve-se \textbf{dobrar os gastos}.
    
    \item \textbf{Criptografia é normalmente contornada, não quebrada.}

    \textit{Adi Shamir é coautor do algoritmo RSA e recebeu o Prêmio Turing em 2002.}
    \begin{figure}
        \centering
        \includegraphics[width=0.7\linewidth]{img/comic_cripto.png}
    \end{figure}
\end{itemize}

\end{frame}


\begin{frame}{Lições Aprendidas}
\vspace{-0.5em}
\begin{itemize}
    \item Nunca, jamais, desenvolva seu próprio algoritmo criptográfico sem ter uma equipe de criptanalistas experientes revisando seu projeto.

    \item Não utilize algoritmos ou protocolos criptográficos que não foram devidamente comprovados.

    \item Atacantes sempre buscam o ponto mais fraco de um sistema criptográfico. Um grande espaço de chaves, por si só, não garante segurança — a cifra pode ser vulnerável a ataques analíticos.

    \item Tamanhos de chave recomendados para algoritmos simétricos, a fim de evitar ataques por busca exaustiva:
    \begin{itemize}
        \item \textbf{64 bits:} inseguro, exceto para dados com valor de curtíssimo prazo.
        \item \textbf{128 bits:} segurança de longo prazo (várias décadas), a menos que computadores quânticos se tornem realidade (o que talvez nunca ocorra).
        \item \textbf{256 bits:} como acima, mas provavelmente seguro mesmo contra ataques quânticos.
    \end{itemize}

    \item A aritmética modular é uma ferramenta elegante para descrever esquemas históricos de encriptação, como a cifra de César, de forma matemática.
\end{itemize}
\end{frame}




\section{Exercícios}

\begin{frame}{Exercício: Segurança de Longo Prazo do AES}

Considere a segurança de longo prazo do AES com chave de 128 bits diante de ataques por busca exaustiva (força bruta).

\begin{enumerate}
    \item Suponha que um atacante possua chips dedicados (\textit{ASICs}) capazes de testar $5 \cdot 10^8$ chaves por segundo. O orçamento total é de \$1 milhão. Cada ASIC custa \$50, e assumimos 100\% de custo adicional para integração (placa, alimentação, refrigeração etc).\\[0.3em]
    $\rightarrow$ Quantos ASICs podem ser utilizados em paralelo?\\
    $\rightarrow$ Qual o tempo médio para encontrar a chave correta?\\
    $\rightarrow$ Compare esse tempo com a idade do Universo ($\approx 10^{10}$ anos).
    
    %\vspace{1em}
    
    \item Agora, considere avanços tecnológicos. A Lei de Moore prevê que o poder computacional dobra a cada 18 meses, enquanto o custo dos circuitos integrados permanece constante.\\[0.3em]
    $\rightarrow$ Quantos anos teremos que esperar até que uma máquina de busca exaustiva possa quebrar o AES-128 em, no máximo, 24 horas?\\
    Assuma novamente um orçamento de \$1 milhão (desconsidere inflação).
\end{enumerate}

\end{frame}

\begin{frame}{\textbf{Solução} (Parte 1/2): Capacidade Atual de Ataque}

\textbf{Dados:}
\begin{itemize}
    \item Orçamento: \$1 milhão
    \item Custo por ASIC (com overhead): \$100
    \item $\Rightarrow$ Número de ASICs: $1\,000\,000 \div 100 = 10\,000$
    \item Velocidade de cada ASIC: $5 \times 10^8$ chaves/s
\end{itemize}

\vspace{0.5em}

\textbf{Capacidade total:}
\[
10\,000 \times 5 \times 10^8 = 5 \times 10^{12} \text{ chaves/s}
\]

\vspace{0.5em}

\textbf{Tempo médio para encontrar a chave correta:}
\[
T = \frac{2^{127}}{5 \times 10^{12}} \approx 4.25 \times 10^{25} \text{ segundos}
\]
\[
\Rightarrow \text{em anos: } \frac{4.25 \times 10^{25}}{60 \times 60 \times 24 \times 365} \approx 1.35 \times 10^{18} \text{ anos}
\]

\vspace{0.5em}
\textbf{Conclusão:} Muito maior que a idade do Universo ($\sim 10^{10}$ anos).

\end{frame}

\begin{frame}{\textbf{Solução} (Parte 2/2): Projeção com a Lei de Moore}

\textbf{Objetivo:}
\begin{itemize}
    \item Reduzir o tempo médio de quebra para no máximo 24 horas ($86400$ segundos).
\end{itemize}

\vspace{0.5em}

\textbf{Cálculo:}
\[
\frac{2^{127}}{V} \leq 86400
\quad \Rightarrow \quad
V \geq \frac{2^{127}}{86400}
\]
\[
V \approx 1.46 \times 10^{34} \text{ chaves/s}
\]

\vspace{0.5em}

\textbf{Comparação com a capacidade atual:}
\[
\text{Velocidade atual: } 5 \times 10^{12} \text{ chaves/s}
\]
\[
\text{Número de duplicações necessárias: } \log_2\left(\frac{1.46 \times 10^{34}}{5 \times 10^{12}}\right) \approx 74
\]

\vspace{0.5em}

\textbf{Tempo estimado considerando a Lei de Moore:}
\[
74 \times 1.5 = \textbf{111 anos}
\]

\end{frame}




\begin{frame}{Exercício: Relação entre Senhas e Tamanho da Chave}

Consideramos agora a relação entre senhas e o tamanho da chave. Para isso, analisamos um sistema criptográfico no qual o usuário insere uma chave na forma de uma senha.

%\vspace{0.8em}
\begin{enumerate}
    \item Suponha uma senha composta por 8 caracteres, onde cada caractere é codificado com ASCII (7 bits por caractere, ou seja, 128 possíveis caracteres).\\
    Qual é o tamanho do espaço de chaves que pode ser construído com essas senhas?

    %\vspace{0.5em}
    \item Qual é o comprimento correspondente da chave em bits?

    %\vspace{0.5em}
    \item Suponha que a maioria dos usuários utilize apenas as 26 letras minúsculas do alfabeto, em vez dos 128 caracteres ASCII.\\
    Qual seria o comprimento correspondente da chave em bits nesse caso?

    %vspace{0.5em}
    \item Quantos caracteres são necessários, no mínimo, para gerar uma chave de 128 bits no caso de:
    \begin{enumerate}
        \item Caracteres de 7 bits?
        \item Letras minúsculas do alfabeto (26 símbolos)?
    \end{enumerate}
\end{enumerate}

\end{frame}

\begin{frame}{\textbf{Solução}: Relação entre Senhas e Tamanho da Chave}

\begin{enumerate}
    \item \textbf{Espaço de chaves com 8 caracteres ASCII (7 bits):}\\
    Cada caractere: $2^7 = 128$ possibilidades.\\
    Total para 8 caracteres: $128^8 = 2^{56}$ combinações.

    \vspace{0.5em}
    \item \textbf{Tamanho da chave em bits:} \\
    $2^{56}$ combinações correspondem a uma chave de \textbf{56 bits}.

    \vspace{0.5em}
    \item \textbf{Se o usuário usa apenas 26 letras minúsculas:} \\
    Espaço de chaves: $26^8$ combinações.\\
    $26^8 \approx 2^{37.6}$, logo a chave tem aproximadamente \textbf{38 bits}.

    \vspace{0.5em}
    \item \textbf{Mínimo de caracteres necessários para atingir 128 bits de segurança:}
    \begin{enumerate}
        \item Para caracteres de 7 bits: \\
        Precisamos de $2^{7n} \geq 2^{128} \Rightarrow n \geq \frac{128}{7} \approx \textbf{18.3} \Rightarrow \textbf{19 caracteres}$

        \item Para letras minúsculas (26 símbolos): \\
        $26^n \geq 2^{128} \Rightarrow n \geq \frac{128}{\log_2 26} \approx \frac{128}{4.7} \approx \textbf{27.2} \Rightarrow \textbf{28 caracteres}$
    \end{enumerate}
\end{enumerate}

\end{frame}


\begin{frame}{Exercício: Cifra de Vigenère}

Consideramos uma extensão da cifra de César: a cifra de \textbf{Vigenère}, que utiliza uma sequência de deslocamentos $k_i$ derivada de uma palavra-código $c = (c_0, c_1, \ldots, c_{l-1})$ com $l$ letras.


Cada letra $c_i$ é mapeada para um número de 0 a 25, baseado em sua posição no alfabeto (A = 0, B = 1, ..., Z = 25). A cifra aplica deslocamentos cíclicos:

\[
y_j \equiv x_j + k_{j \bmod l} \mod 26
\]


\textbf{1.} Dada a palavra-código \texttt{JAMAIKA} ($l = 7$), converta as letras nos valores $k_i$ correspondentes.\\


\textbf{2.} Utilize a tabela de substituição (alfabeto deslocado) para cifrar a palavra \texttt{CODEBREAKERS}.\\
Para cada letra, aplique o deslocamento $k_{j \bmod l}$ correspondente.


\textbf{3.} Analise a segurança da cifra de Vigenère.\\
Você consegue imaginar um ataque contra ela? Como descobrir a palavra-código?

\end{frame}

\begin{frame}{Tabela de Deslocamento para Cifra Polialfabética}

\begin{figure}
        \centering
        \includegraphics[width=0.7\linewidth]{img/polyalphabetic.png}
\end{figure}
    
\end{frame}

\end{document}
