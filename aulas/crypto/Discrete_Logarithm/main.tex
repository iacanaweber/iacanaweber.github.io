\documentclass{beamer}



\usepackage{amsmath}
\usepackage{booktabs} % Para tabelas com visual profissional
\usepackage{tabularx} % Para tabelas com largura definida e colunas do tipo X


\usetheme{Madrid}
\usecolortheme{beaver}
\usepackage[utf8]{inputenc}
\usepackage[brazil]{babel}
\usepackage{graphicx}
\usepackage{xcolor}
\usepackage{colortbl} % para sombrear células da tabela
\usepackage{tikz}
\usetikzlibrary{shapes, arrows.meta, positioning}
\usepackage{array} 

\usepackage{tcolorbox}     % Para caixas coloridas e formatadas
\tcbuselibrary{skins, breakable}  % Habilita opções avançadas do tcolorbox
\usepackage{graphicx}      % Necessário para \scalebox
\usepackage{hyperref}
\hypersetup{
    colorlinks=true,
    %linkcolor=blue,
    filecolor=magenta,      
    urlcolor=cyan,
    pdfpagemode=FullScreen,
    }

\urlstyle{same}


\definecolor{graycell}{gray}{0.85} % define a cor de fundo da célula
\definecolor{vermelhoClaro}{RGB}{255, 200, 200}


\AtBeginSection[]
{
  \begin{frame}
    \frametitle{Índice}
    \tableofcontents[currentsection]
  \end{frame}
}

% Informações da apresentação
\title[RSA]{RSA}
\author{Prof. Iaçanã Ianiski Weber}
\institute{PUCRS}
\date{CSS (98G08-4)}

\begin{document}

% Slide de Capa Personalizado
\begin{frame}[plain]

% Linha superior com logo e título
\noindent
\begin{minipage}[t]{0.15\textwidth}
    \includegraphics[width=\linewidth]{img/Brasão_PUCRS.png}
\end{minipage}%
\begin{minipage}[t]{0.84\textwidth}
    \begin{center}
        \raisebox{2ex}{\LARGE\bfseries\textcolor{blue} {\textbf{Criptosistemas de Chave Pública}}}\\
        \raisebox{2ex}{\LARGE\bfseries\textcolor{blue}{Baseados no Problema do}}\\
        \raisebox{2ex}{\LARGE\bfseries\textcolor{blue}{Logaritmo Discreto}}\\
    \end{center}
\end{minipage}

\vspace{2cm}

\begin{center}
    \Large \textbf{Prof. Dr. Iaçanã Ianiski Weber} \\
    \vspace{0.2cm}
    \normalsize \textit{Confiabilidade e Segurança de Software} \\
    \vspace{0.2cm}
    \large 98G08-4
\end{center}

\vfill

\begin{center}
    \tiny \textit{Agradecimentos especiais ao Prof. Avelino Zorzo e aos Autores Christof Paar e Jan Pelzl pelo material.}
\end{center}

\end{frame}

% Slide de Índice
\begin{frame}{Índice}
    \tableofcontents
\end{frame}

%%%%%%%%%%%%%%%%%%%%%%%%%%%%%%%%%%%
%%%%%%%%%%%%%%%%%%%%%%%%%%%%%%%%%%%
%%%%%%%%%%%%%%%%%%%%%%%%%%%%%%%%%%%
\section{Troca de Chaves de Diffie–Hellman}

\begin{frame}{Troca de Chaves Diffie-Hellman: Visão Geral}
    \begin{itemize}
        \item Proposto em 1976 por Whitfield Diffie e Martin Hellman
        \vspace{0.5em}
        \item Amplamente utilizado, e.g., em Secure Shell (SSH), Transport Layer Security (TLS) e Internet Protocol Security (IPSec)
        \vspace{0.5em}
        \item O Diffie-Hellman Key Exchange (DHKE) é um protocolo de troca de chaves e \textbf{não} utilizado para criptografia
        
        \item[] (Para fins de criptografia baseada no DHKE, o ElGamal pode ser utilizado.)
    \end{itemize}
\end{frame}

\begin{frame}{Troca de Chaves Diffie-Hellman: Configuração Inicial}
    \begin{enumerate}
        \item Escolha um primo grande $p$.
        \vspace{1em}
        \item Escolha um inteiro $\alpha \in \{2, 3, \ldots, p-2\}$.
        \vspace{1em}
        \item Publique $p$ e $\alpha$.
    \end{enumerate}
\end{frame}

\begin{frame}{Troca de Chaves Diffie-Hellman: Protocolo}
\centering
\vspace*{-0.5em} % Reduce top margin a bit if necessary
\small % Use smaller font for the table content

% The \resizebox command will scale the table to \linewidth.
% \usepackage{graphicx} is required for \resizebox.
\resizebox{\linewidth}{!}{%
\begin{tabular}{p{0.42\textwidth} c p{0.42\textwidth}}
    \multicolumn{1}{c}{\textbf{Alice}} & & \multicolumn{1}{c}{\textbf{Bob}} \\
    \\[1ex]
    Escolhe key privada aleatória \newline $k_{prA} = a \in \{1, 2, \ldots, p-1\}$ & & Escolhe key privada aleatória \newline $k_{prB} = b \in \{1, 2, \ldots, p-1\}$ \\
    \\[2ex]
    Calcula key pública correspondente \newline $k_{pubA} = A = \alpha^a \text{ mod } p$ & $\xrightarrow{\hspace*{2em}A\hspace*{2em}}$ & \\
    \\[2ex]
    & $\xleftarrow{\hspace*{2em}B\hspace*{2em}}$ & Calcula key pública correspondente \newline $k_{pubB} = B = \alpha^b \text{ mod } p$ \\
    \\[2ex]
    Calcula segredo comum \newline $k_{AB} = B^a = (\alpha^b)^a \text{ mod } p$ & & Calcula segredo comum \newline $k_{AB} = A^b = (\alpha^a)^b \text{ mod } p$ \\
\end{tabular}%
} % End of \resizebox

\centerline{\makebox[0.85\linewidth]{\dotfill}} % Dotted 
\vspace{-0.3em}

\begin{center}
    Podemos agora usar a key conjunta $k_{AB}$ para criptografia, e.g., com AES: \\[1em]
    $y = \text{AES}_{k_{AB}}(x) \qquad \xrightarrow{\hspace*{2em} y \hspace*{2em}} \qquad x = \text{AES}_{k_{AB}}^{-1}(y)$
\end{center}
\end{frame}

\begin{frame}{Troca de Chaves Diffie-Hellman: Exemplo}
    \centering
    \vspace*{-0.5em} % Adjust top margin if needed
    \tiny{Parâmetros: $p=29, \alpha=2$}
    \small % Use smaller font for the table content

    % The \resizebox command will scale the table to \linewidth.
    % \usepackage{graphicx} is required for \resizebox.
    \resizebox{\linewidth}{!}{%
    \begin{tabular}{p{0.42\textwidth} c p{0.42\textwidth}}
        \multicolumn{1}{c}{\textbf{Alice}} & & \multicolumn{1}{c}{\textbf{Bob}} \\
        \\
        Escolhe key privada aleatória \newline $k_{prA} = a = 5$ & & Escolhe key privada aleatória \newline $k_{prB} = b = 12$ \\
        \\[1ex]
        Calcula key pública correspondente \newline $k_{pubA} = A = 2^5 = 3 \text{ mod } 29$ & $\xrightarrow{\hspace*{2em}3\hspace*{2em}}$ & \\
        \\[1ex]
        & $\xleftarrow{\hspace*{2em}7\hspace*{2em}}$ & Calcula key pública correspondente \newline $k_{pubB} = B = 2^{12} = 7 \text{ mod } 29$ \\
        \\[1ex]
        Calcula segredo comum \newline $k_{AB} = B^a = 7^5 = 16 \text{ mod } 29$ & & Calcula segredo comum \newline $k_{AB} = A^b = 3^{12} = 16 \text{ mod } 29$ \\
    \end{tabular}%
    } % End of \resizebox

    \begin{minipage}{0.9\linewidth} % To center this block and allow indentation
    \vspace{0.2em}
    \begin{center}
    \textbf{Prova de correção} \par    
    Alice calcula: $B^a = (\alpha^b)^a \text{ mod } p$ \par
    Bob calcula: $A^b = (\alpha^a)^b \text{ mod } p$ \par
    Ou seja, Alice e Bob calculam a mesma key $k_{AB}$!
    \end{center}
    \end{minipage}

\end{frame}

\section{O Problema do Logaritmo Discreto}
\begin{frame}{O Problema do Logaritmo Discreto}
    \small Problema do Logaritmo Discreto (\textit{Discrete Logarithm Problem} - DLP) em $\mathbb{Z}_p^*$.
    \normalsize
    \begin{itemize}
        \item Dado o grupo cíclico finito $\mathbb{Z}_p^*$ de ordem $p-1$, um elemento primitivo $\alpha \in \mathbb{Z}_p^*$ e outro elemento $\beta \in \mathbb{Z}_p^*$.
        \item O DLP é o problema de determinar o inteiro $x$ tal que $1 \le x \le p-1$ e $\alpha^x \equiv \beta \text{ mod } p$.
        \item Esta computação é chamada de \textbf{problema do logaritmo discreto}.
    \end{itemize}

    \begin{center}
    $x = \log_{\alpha} \beta \text{ mod } p$
    \end{center}

    \begin{itemize}
        \item Exemplo: Calcule $x$ para $5^x \equiv 41 \text{ mod } 47$.
    \end{itemize}
\end{frame}

\begin{frame}{O Problema do Logaritmo Discreto Generalizado}
    \begin{itemize}
        \item Dado um grupo cíclico finito $G$ com a operação de grupo $\circ$ e cardinalidade $n$.
        \vspace{0.5em}
        \item Consideramos um elemento primitivo $\alpha \in G$ e outro elemento $\beta \in G$.
        \vspace{0.5em}
        \item O problema do logaritmo discreto consiste em encontrar o inteiro $x$, onde $1 \le x \le n$, tal que:
    \end{itemize}
    \[
        \beta = \underbrace{\alpha \circ \alpha \circ \ldots \circ \alpha}_{x \text{ vezes}} = \alpha^x
    \]
\end{frame}

\begin{frame}{O Problema do Logaritmo Discreto Generalizado}
    \small % Use a slightly smaller font for better fit
    Os seguintes problemas de logaritmo discreto foram propostos para uso em criptografia:

    \begin{enumerate}
        \item O grupo multiplicativo do corpo primo $\mathbb{Z}_p$ ou um subgrupo dele. Por exemplo, o DHKE clássico utiliza este grupo (slides anteriores), mas também a criptografia Elgamal ou o Digital Signature Algorithm (DSA).
        \vspace{0.5em}
        \item O grupo cíclico formado por uma curva elíptica.
        \vspace{0.5em}
        \item O grupo multiplicativo de um corpo de Galois $GF(2^m)$ ou um subgrupo dele. Esquemas como o DHKE podem ser realizados com eles.
        \vspace{0.5em}
        \item Curvas hiperelípticas ou variedades algébricas, que podem ser vistas como generalizações de curvas elípticas.
    \end{enumerate}

    \vspace{1em}
    \textbf{Observação:} Os grupos 1. e 2. são os mais frequentemente utilizados na prática.
\end{frame}

\section{Segurança da Troca de Chaves Diffie-Hellman}

\begin{frame}{Ataques contra o Problema do Logaritmo Discreto}
    \small % Para melhor ajuste do conteúdo no slide
    \begin{itemize}
        \item A segurança de muitas primitivas assimétricas baseia-se na dificuldade de calcular o DLP em grupos cíclicos, i.e., calcular $x$ para dados $\alpha$ e $\beta$ tais que:
        \[
            \beta = \underbrace{\alpha \circ \alpha \circ \ldots \circ \alpha}_{x \text{ vezes}} = \alpha^x
        \]
        \item Existem os seguintes algoritmos para calcular logaritmos discretos:
        \begin{itemize}
            \item Algoritmos genéricos: Funcionam em qualquer grupo cíclico
            \begin{itemize}
                \item Busca por Força Bruta
                \item Método Baby-Step-Giant-Step de Shanks
                \item Método Rho de Pollard
                \item Método de Pohlig-Hellman
            \end{itemize}
            \item Algoritmos não genéricos: Funcionam apenas em grupos específicos, em particular em $\mathbb{Z}_p$
            \begin{itemize}
                \item Método do Cálculo de Índices
            \end{itemize}
        \end{itemize}
        \item \textbf{Observação:} Curvas elípticas só podem ser atacadas com algoritmos genéricos, que são mais fracos que os algoritmos não genéricos. Portanto, curvas elípticas são seguras com comprimentos de key menores do que o DLP em corpos primos $\mathbb{Z}_p$.
    \end{itemize}
\end{frame}

\begin{frame}{Ataques contra o Problema do Logaritmo Discreto}
    \small % Para melhor ajuste do conteúdo no slide

    Resumo dos recordes para o cálculo de logaritmos discretos em $\mathbb{Z}_p^*$:

    \vspace{0.5em}
    \begin{center}
    \begin{tabular}{|c|c|c|}
        \hline
        \textbf{Dígitos decimais} & \textbf{Comprimento em bits} & \textbf{Data} \\
        \hline
        58  & 193 & 1991 \\
        68  & 216 & 1996 \\
        85  & 282 & 1998 \\
        100 & 332 & 1999 \\
        120 & 399 & 2001 \\
        135 & 448 & 2006 \\
        160 & 532 & 2007 \\
        232 & 768 & 2016 \\ % Dado atualizado
        240 & 795 & 2019 \\ % Dado atualizado
        \hline
    \end{tabular}
    \end{center}
    \vspace{1em}

    A fim de prevenir ataques que calculam o DLP, recomenda-se usar primos com um comprimento de pelo menos 2048 bits para esquemas como o Diffie-Hellman em $\mathbb{Z}_p^*$.

\end{frame}

\begin{frame}[plain] % Use [plain] to remove Beamer's header/footer for more space
    \frametitle{Ataques contra o Problema do Logaritmo Discreto} % Optional: Keep title if needed
    \begin{figure}
        \centering
        \includegraphics[width=\textwidth, height=0.8\textheight, keepaspectratio]{img/dlp_evolution_quadratic_trend_graph.png}
    \end{figure}
\end{frame}

\begin{frame}{Segurança da Troca de Chaves Diffie-Hellman Clássica}
    \small % Para melhor ajuste do conteúdo no slide
    \begin{itemize}
        \item Quais informações Oscar possui?
        \begin{itemize}
            \item $\alpha, p$
            \item $k_{pubA} = A = \alpha^a \text{ mod } p$
            \item $k_{pubB} = B = \alpha^b \text{ mod } p$
        \end{itemize}
        \item Quais informações Oscar quer obter?
        \begin{itemize}
            \item $k_{AB} = \alpha^{ba} = \alpha^{ab} \text{ mod } p$
            \item Isto é conhecido como Problema de Diffie-Hellman (DHP).
        \end{itemize}
        \item A única forma conhecida de resolver o DHP é resolver o DLP, i.e.,
        \begin{enumerate}
            \item Calcular $a = \log_{\alpha} A \text{ mod } p$.
            \item Calcular $k_{AB} = B^a = \alpha^{ba} \text{ mod } p$.
        \end{enumerate}
        Conjectura-se que o DHP e o DLP são equivalentes, ou seja, resolver o DHP implica resolver o DLP.
        \item Para prevenir ataques, ou seja, para prevenir que o DLP possa ser resolvido, escolha $p > 2^{1024}$.
    \end{itemize}
\end{frame}

\begin{frame}{Lições Aprendidas}
    \begin{itemize}
        \item O protocolo Diffie-Hellman é um método amplamente utilizado para troca de chaves. Ele é baseado em grupos cíclicos.
        \item O problema do logaritmo discreto é uma das mais importantes funções de mão única na criptografia assimétrica moderna. Muitos algoritmos de chave pública são baseados nele.
        \item Para o protocolo Diffie-Hellman em $\mathbb{Z}_p^*$, \textit{o primo p deve ter pelo menos 1024 bits} de comprimento. Isto proporciona uma segurança aproximadamente equivalente a uma cifra simétrica de 80 bits.
        \item Para uma melhor segurança a longo prazo, um primo de comprimento 2048 bits deve ser escolhido.
    \end{itemize}
\end{frame}

\section{Exercícios}

\begin{frame}{Exercício 1: Cálculo de Chaves DHKE}
    Calcule as duas chaves públicas e a chave conjunta $k_{AB}$ para o esquema DHKE com os parâmetros $p=467, \alpha=2$, e os seguintes pares de chaves privadas $(a,b)$:

    \begin{itemize}
        \item $a=3, b=5$
        \item $a=400, b=134$
        \item $a=228, b=57$
    \end{itemize}
        
    \vspace{1em}

    Em todos os casos, realize o cálculo da chave conjunta tanto do ponto de vista de Alice quanto de Bob. Isso servirá como uma verificação dos seus resultados.
\end{frame}

\begin{frame}{Exercício 2: Chaves Privadas no DHKE}
    \small
    No protocolo DHKE, as chaves privadas são escolhidas do conjunto
    \[ \{2, \ldots, p-2\} \]
    \vspace{1em}
    Por que os valores 1 e $p-1$ são excluídos? Descreva a fraqueza desses dois valores.
\end{frame}

\begin{frame}{Explicação: Fraqueza das Chaves Privadas 1 e $p-1$}
    \small
    A exclusão dos valores 1 e $p-1$ para chaves privadas no protocolo Diffie-Hellman (DHKE) é crucial para a segurança. Vejamos as fraquezas associadas a cada um:

    \vspace{1em}
    \textbf{Caso 1: Chave Privada $a=1$}
    \begin{itemize}
        \item \textbf{Chave Pública de Alice ($A$):} Se Alice escolhe $a=1$, sua chave pública se torna:
        \[ A = \alpha^a \pmod p = \alpha^1 \pmod p = \alpha \pmod p \]
        \item \textbf{Chave Conjunta ($K_{AB}$):} A chave conjunta calculada por Alice seria:
        \[ K_{AB} = B^a \pmod p = B^1 \pmod p = B \pmod p \]
        \item \textbf{Fraqueza:} A chave conjunta é simplesmente a chave pública de Bob ($B$). Se um invasor (Oscar) interceptar $B$ (que é transmitido publicamente), ele conhecerá diretamente a chave secreta $K_{AB}$. Alice não adiciona nenhuma entropia ou segurança à chave.
    \end{itemize}

\end{frame}

\begin{frame}{Explicação: Fraqueza das Chaves Privadas 1 e $p-1$}
    
    \textbf{Caso 2: Chave Privada $a=p-1$}
    \begin{itemize}
        \item \textbf{Contexto (Pequeno Teorema de Fermat):} Se $p$ é um número primo, então para qualquer inteiro $\alpha$ não divisível por $p$, temos $\alpha^{p-1} \equiv 1 \pmod p$.
        \item \textbf{Chave Pública de Alice ($A$):} Se Alice escolhe $a=p-1$, sua chave pública se torna:
        \[ A = \alpha^a \pmod p = \alpha^{p-1} \pmod p \equiv 1 \pmod p \]
        \item \textbf{Chave Conjunta ($K_{AB}$):} A chave conjunta calculada por Alice seria:
        \[ K_{AB} = B^a \pmod p = B^{p-1} \pmod p \]
        Considerando que $B = \alpha^b \pmod p$ e $B \not\equiv 0 \pmod p$, então $B^{p-1} \equiv 1 \pmod p$.
        \[ K_{AB} \equiv 1 \pmod p \]
        \item \textbf{Fraqueza:} A chave conjunta resultante é sempre 1. Uma chave secreta de valor 1 é trivial, previsível e não oferece segurança alguma.
    \end{itemize}

\end{frame}

\end{document}
