\documentclass{beamer}

\usetheme{Madrid}
\usecolortheme{beaver}

\usepackage[utf8]{inputenc}
\usepackage[brazil]{babel}
\usepackage{amsmath, amssymb}
\usepackage{booktabs}
\usepackage{array}
\usepackage{graphicx}
\usepackage{tcolorbox}
\usepackage{hyperref}
\usepackage{tikz}
\usetikzlibrary{arrows.meta, positioning, shapes.geometric}

\tcbuselibrary{skins, breakable}

\AtBeginSection[]
{
  \begin{frame}
    \frametitle{Índice}
    \tableofcontents[currentsection]
  \end{frame}
}

% --------- Macros úteis ----------
\newcommand{\R}{\mathrm{R}}
\newcommand{\A}{\mathrm{A}}
\newcommand{\F}{\mathrm{F}}
\newcommand{\fTTF}{\mathrm{f}}
\newcommand{\haz}{\mathrm{h}}
\newcommand{\MTTF}{\mathrm{MTTF}}
\newcommand{\MTBF}{\mathrm{MTBF}}
\newcommand{\MTTR}{\mathrm{MTTR}}
\newcommand{\MDT}{\mathrm{MDT}}
\newcommand{\TTT}{\mathrm{TTT}} % total time on test
\newcommand{\SLO}{\mathrm{SLO}}
\newcommand{\SLA}{\mathrm{SLA}}
\newcommand{\SLI}{\mathrm{SLI}}

\title[Dependabilidade: Confiabilidade e Disponibilidade]{Dependabilidade: Confiabilidade e Disponibilidade}
\subtitle{Modelos, Métricas, Composição de Sistemas e Práticas de Engenharia}
\author{Prof. Iaçanã Ianiski Weber}
\institute{PUCRS}
\date{CSS (98G08-4) -- Aula 2}

\begin{document}

% ---------------- Capa (mantida) ----------------
\begin{frame}[plain]
\noindent
\begin{minipage}[t]{0.15\textwidth}
    \includegraphics[width=\linewidth]{../img/brasao_pucrs.png}
\end{minipage}%
\begin{minipage}[t]{0.75\textwidth}
    \begin{center}
        \raisebox{5ex}{\LARGE\bfseries\textcolor{blue}{\textbf{Dependabilidade:}}}\\
        \raisebox{2ex}{\LARGE\bfseries\textcolor{blue}{Confiabilidade e Disponibilidade}}\\
    \end{center}
\end{minipage}

\vspace{1.4cm}

\begin{center}
    \Large \textbf{Prof. Dr. Iaçanã Ianiski Weber} \\
    \vspace{0.2cm}
    \normalsize \textit{Confiabilidade e Segurança de Software} \\
    \vspace{0.2cm}
    \large 98G08-4
\end{center}

\vfill
\begin{center}
    \tiny \textit{Gancho para a próxima aula: Safety, Security e Resiliência (atributos e ameaças ampliadas).}
\end{center}
\end{frame}

\begin{frame}{Índice}
  \tableofcontents
\end{frame}

% =========================================================
\section{Motivação e Taxonomia de Dependabilidade}

\begin{frame}{Por que engenharia de dependabilidade?}
\begin{itemize}
  \item Sistemas reais falham por \textbf{causas acidentais} e \textbf{causas sistemáticas} (projeto/implementação).
  \item Em ambientes modernos: \textbf{dependências} (serviços, bibliotecas, redes) dominam risco.
  \item Engenharia precisa de:
    \begin{itemize}
      \item \textbf{Métricas} (quantificar), \textbf{modelos} (predizer/planejar),
      \item \textbf{métodos} (prevenir, tolerar, remover, prever falhas).
    \end{itemize}
\end{itemize}
% TALKING POINT: contraste entre "não falhar" (reliability) e "falhar pouco e recuperar rápido" (availability).
% TODO: inserir exemplo real (case de outage conhecido) e discutir dependências.
\end{frame}

\begin{frame}{Dependabilidade (Dependability): definição e atributos}
\begin{block}{Definição (visão clássica)}
Dependabilidade é a \textbf{capacidade de entregar serviço correto e confiável} em que se pode \textbf{justificadamente confiar}.
\end{block}
\begin{itemize}
  \item Atributos (exemplos): \textbf{confiabilidade}, \textbf{disponibilidade}, \textbf{segurança funcional (safety)}, \textbf{integridade}, \textbf{manutenibilidade}.
  \item \textbf{Security} adiciona preocupações como \textbf{confidencialidade} além de disponibilidade/integridade.
\end{itemize}
\small
\hfill\footnotesize{Fonte clássica: Avizienis et al., IEEE TDSC (2004).}
% TODO: link (PDF): https://www.landwehr.org/2004-aviz-laprie-randell.pdf
\end{frame}

\begin{frame}{A cadeia: Fault \texorpdfstring{$\to$}{->} Error \texorpdfstring{$\to$}{->} Failure}
\begin{columns}[T,onlytextwidth]
\column{0.60\textwidth}
\begin{itemize}
  \item \textbf{Fault (defeito)}: causa interna/externa latente.
  \item \textbf{Error (erro de estado)}: estado interno incorreto gerado por fault ativado.
  \item \textbf{Failure (falha de serviço)}: erro chega à interface de serviço (desvio do especificado).
\end{itemize}

\column{0.40\textwidth}
\centering
\begin{tikzpicture}[
  node distance=6mm,
  box/.style={draw, rounded corners, align=center, inner sep=3mm},
  arr/.style={-Latex, thick}
]
\node[box] (fault) {Fault\\(defeito)};
\node[box, below=of fault] (error) {Error\\(estado incorreto)};
\node[box, below=of error] (failure) {Failure\\(serviço incorreto)};
\draw[arr] (fault) -- (error);
\draw[arr] (error) -- (failure);
\end{tikzpicture}
\end{columns}
% TALKING POINT: em Security, faults podem ser "maliciosamente injetados" (ataques) e não apenas acidentais.
% TODO: referência de slide universitário: Univ. of Pisa (dep.) menciona explicitamente a cadeia.
% TODO link: https://docenti.ing.unipi.it/~a008669/didattica/ANNO2015-16/DEP/BasicConceptsP3.pdf
\end{frame}

\begin{frame}{Reliability vs Availability: intuição correta}
\begin{block}{Confiabilidade (Reliability)}
Probabilidade de \textbf{não ocorrer falha} durante um intervalo: \(\R(t) = P(T>t)\).
\end{block}
\begin{block}{Disponibilidade (Availability)}
Probabilidade de o sistema estar \textbf{operacional em um instante} ou em regime estacionário.
\end{block}

\begin{itemize}
  \item Um sistema pode ser \textbf{pouco confiável} (falha com frequência) mas \textbf{muito disponível} (recupera rápido).
  \item E o inverso: falha raramente, mas quando falha, fica indisponível por muito tempo.
\end{itemize}
% TALKING POINT: discutir exemplos (serviço web com autoscaling/rollback vs equipamento com reparo demorado).
\end{frame}

% =========================================================
\section{Fundamentos Probabilísticos de Confiabilidade}

\begin{frame}{TTF como variável aleatória}
Seja \(T\) o \textbf{tempo até falha} (Time-To-Failure).
\[
\F(t)=P(T\le t), \quad \R(t)=P(T>t)=1-\F(t)
\]
Densidade:
\[
\fTTF(t)=\frac{d\F(t)}{dt}
\]
\textbf{Hazard rate} (taxa de risco/instantânea):
\[
\haz(t)=\frac{\fTTF(t)}{\R(t)}
\]
% TALKING POINT: hazard é "taxa condicional de falhar agora dado que sobreviveu até t".
% TODO: inserir figura intuitiva de hazard (ex.: gráfico h(t) para Weibull beta<1,=1,>1).
\end{frame}

\begin{frame}{MTTF como área sob \texorpdfstring{$\R(t)$}{R(t)}}
Para distribuições contínuas (não negativas):
\[
\MTTF = \mathbb{E}[T] = \int_0^{\infty} \R(t)\, dt
\]
\begin{itemize}
  \item Esse resultado é muito útil: você não precisa da densidade explicitamente para obter \(\MTTF\).
  \item Para modelos paramétricos, \(\MTTF\) vira função direta de parâmetros (ex.: \(1/\lambda\)).
\end{itemize}
% TALKING POINT: mostrar derivação curta em sala (integração por partes).
\end{frame}

\begin{frame}{Modelo exponencial: quando faz sentido?}
\begin{block}{Hipótese-chave}
\(\haz(t)=\lambda\) constante \(\Rightarrow\) \(\R(t)=e^{-\lambda t}\) (propriedade sem memória).
\end{block}
\begin{itemize}
  \item Bom modelo \textbf{na região “útil”} da curva \textbf{bathtub} (taxa aproximadamente constante).
  \item Muito usado em testes/planejamento por simplicidade (e por permitir ICs analíticos).
\end{itemize}
\small
\hfill\footnotesize{Referência prática: NIST Reliability/Availability Handbook (exponencial/HPP).}
% TODO link NIST: https://www.itl.nist.gov/div898/handbook/apr/section4/apr451.htm
\end{frame}

\begin{frame}{Exponencial: métricas e estimação rápida}
\[
\R(t)=e^{-\lambda t},\quad \MTTF=\frac{1}{\lambda}
\]
Se em um teste observamos \(n\) falhas e \(\TTT\) = \textbf{tempo total em teste} (soma das exposições):
\[
\hat{\lambda}=\frac{n}{\TTT},\quad \widehat{\MTTF}=\frac{\TTT}{n}
\]
\begin{itemize}
  \item \(\TTT\) funciona bem inclusive com \textbf{censura à direita} (itens que não falharam até o fim do teste).
\end{itemize}
% TALKING POINT: reforçar a diferença entre "contar falhas" e "tempo total de exposição".
\end{frame}

\begin{frame}{Intervalo de confiança para MTBF/MTTF (exponencial)}
No modelo exponencial (HPP), o NIST fornece fatores/tabelas e forma fechada para IC (baseado em \(\chi^2\)).
\begin{itemize}
  \item Essencial para não vender “um número mágico” sem incerteza.
  \item Caso “zero falhas” (muito comum em testes curtos) tem tratamento específico.
\end{itemize}
\small
\hfill\footnotesize{Fonte: NIST APR 8.4.5.1 (MTBF e ICs).}
% TODO link NIST: https://www.itl.nist.gov/div898/handbook/apr/section4/apr451.htm
% TALKING POINT: em sala, fazer um mini-exemplo de "0 falhas em X horas" -> bound inferior de MTBF com 90% de confiança.
\end{frame}

\begin{frame}{Weibull: modelo mais realista para componentes físicos}
\begin{block}{Weibull (2 parâmetros)}
\[
\R(t)=\exp\left(-\left(\frac{t}{\eta}\right)^{\beta}\right),
\quad
\haz(t)=\frac{\beta}{\eta}\left(\frac{t}{\eta}\right)^{\beta-1}
\]
\end{block}
\begin{itemize}
  \item \(\beta<1\): “infant mortality” (hazard decrescente).
  \item \(\beta=1\): exponencial (hazard constante).
  \item \(\beta>1\): desgaste/envelhecimento (hazard crescente).
\end{itemize}
\small
\hfill\footnotesize{Referência didática: ETH Zürich (Reliability Distributions).}
% TODO link ETH: https://www.ises.ethz.ch/content/dam/ethz/special-interest/mavt/imes/ises-dam/documents/Slides_Reliability_of_Technical_Systems/02-Reliability-Distributions.pdf
\end{frame}

\begin{frame}{Bathtub curve e interpretação de engenharia}
\begin{itemize}
  \item Três fases típicas:
    \begin{enumerate}
      \item \textbf{Mortalidade infantil}: falhas de fabricação/integração (hazard decrescente).
      \item \textbf{Vida útil}: falhas aleatórias (hazard ~ constante).
      \item \textbf{Desgaste}: envelhecimento (hazard crescente).
    \end{enumerate}
  \item Em \textbf{software}, a analogia muda:
    \begin{itemize}
      \item Sem mudanças: hazard pode ser aproximadamente constante.
      \item Com correções/novas features: hazard varia e pode até piorar (regressões).
    \end{itemize}
\end{itemize}
% TODO: inserir figura clássica bathtub (NIST/curso universitário).
% TODO link (exemplo): material MIT/notes menciona bathtub e região útil.
% TODO link: https://web.mit.edu/2.611/www/notes/BB.pdf
% TALKING POINT: relacionar com "burn-in", testes de aceitação, e com práticas de release/rollback em software.
\end{frame}

% =========================================================
\section{Disponibilidade: sistemas reparáveis e operação}

\begin{frame}{MTTF, MTBF, MTTR, MDT (e por que MTTR é “decomponível”)}
\begin{itemize}
  \item \(\MTTF\): tempo médio até falha (não reparável).
  \item \(\MTBF\): tempo médio entre falhas (reparável).
  \item \(\MTTR\): tempo médio de reparo/recuperação.
  \item \(\MDT\): downtime médio (pode incluir logística, espera, janela, aprovação).
\end{itemize}
\begin{block}{Decomposição útil (Ops/SRE)}
\[
\MTTR \approx \underbrace{\mathrm{MTTD}}_{\text{detectar}}
+
\underbrace{\mathrm{MTTI}}_{\text{diagnosticar/decidir}}
+
\underbrace{\mathrm{MTTM}}_{\text{mitigar/reparar}}
\]
\end{block}
% TALKING POINT: mostrar como observabilidade reduz MTTD e automação reduz MTTM.
\end{frame}

\begin{frame}{Disponibilidade: instantânea vs estacionária}
\begin{itemize}
  \item \textbf{Disponibilidade instantânea}: \(\A(t)=P(\text{sistema operacional em } t)\).
  \item \textbf{Disponibilidade estacionária}: \(\A_{\infty}=\lim_{t\to\infty}\A(t)\).
\end{itemize}
Para taxas exponenciais: falha \(\lambda\) e reparo \(\mu\),
\[
\A_{\infty}=\frac{\mu}{\lambda+\mu}
\quad\Longleftrightarrow\quad
\A_{\infty}=\frac{\MTBF}{\MTBF+\MTTR}
\]
% TALKING POINT: reforçar que a fórmula MTBF/(MTBF+MTTR) é um caso particular (assume expo/estacionário).
% TODO: link ETH (Markov/availability) para embasamento.
\end{frame}

\begin{frame}{“N-nines” e orçamento de indisponibilidade}
\small
\centering
\begin{tabular}{lcc}
\toprule
Meta (\%) & Downtime/mês (30d) & Downtime/ano (365d) \\
\midrule
99.0   & \(\approx\) 7h12m & \(\approx\) 3d15h \\
99.9   & \(\approx\) 43m12s & \(\approx\) 8h45m \\
99.99  & \(\approx\) 4m19s & \(\approx\) 52m \\
99.999 & \(\approx\) 26s   & \(\approx\) 5m15s \\
\bottomrule
\end{tabular}

\vspace{2mm}
\begin{itemize}
  \item A conta é direta: \(\text{downtime} = (1-\text{meta})\times \text{janela}\).
\end{itemize}
\footnotesize{Referência prática (error budget / janela mensal): Google SRE (Calculus of Service Availability).}
% TODO link: https://sre.google/static/pdf/calculus_of.pdf
% TALKING POINT: discutir diferença entre "calendar month" e "rolling window"; e entre downtime e erros de requisição (SLI).
\end{frame}

\begin{frame}{Disponibilidade percebida vs interna (o que você mede?)}
\begin{itemize}
  \item \textbf{Disponibilidade interna}: processo/host “up”.
  \item \textbf{Disponibilidade do usuário}: requisições boas / total (\(\SLI\)).
  \item \textbf{Erro parcial}: degradação (latência alta, funcionalidade limitada) pode ser “falha” do ponto de vista do SLO.
\end{itemize}
\begin{block}{Boa prática}
Definir SLIs alinhados ao usuário e derivar \(\SLO\) e “error budget” (1-\(\SLO\)).
\end{block}
\footnotesize{Referência: Google SRE Workbook (alerting e SLOs).}
% TODO link: https://sre.google/workbook/alerting-on-slos/
% TALKING POINT: exemplo de SLI (p99 latência + taxa de sucesso) e como isso se conecta com disponibilidade.
\end{frame}

% =========================================================
\section{Composição de Sistemas: RBD, redundância e dependências}

\begin{frame}{Assunção de independência: útil, mas perigosa}
\begin{itemize}
  \item Muitas fórmulas clássicas assumem falhas \textbf{independentes}.
  \item Na prática, falhas correlacionadas são comuns:
    \begin{itemize}
      \item bug em código compartilhado,
      \item configuração errada propagada,
      \item dependência externa (DNS, IdP, rede),
      \item energia/temperatura/ambiente (hardware),
      \item ataques (Security).
    \end{itemize}
\end{itemize}
% TALKING POINT: introduzir "common-cause failures" e por que diversidade (design diversity) pode ser necessária.
% TODO: link ETH (Dependent Failures / Common Cause).
% TODO link: https://www.ises.ethz.ch/content/dam/ethz/special-interest/mavt/imes/ises-dam/documents/Slides_Reliability_of_Technical_Systems/07-Dependent-Failures.pdf
\end{frame}

\begin{frame}{RBD: sistema em série (dependências “AND”)}
Se um serviço depende de \(n\) componentes independentes:
\[
\R_{sys}(t)=\prod_{i=1}^{n} \R_i(t)
\]
\begin{itemize}
  \item Intuição: “uma falha derruba o serviço”.
  \item Em microsserviços, dependências críticas podem dominar a confiabilidade global.
\end{itemize}
% TALKING POINT: relacionar com frase do Google SRE "you're only as available as your dependencies".
\end{frame}

\begin{frame}{RBD: paralelo ativo (redundância “OR”)}
Para \(n\) componentes em paralelo ativo (1-out-of-n):
\[
\R_{sys}(t)=1-\prod_{i=1}^{n}\bigl(1-\R_i(t)\bigr)
\]
Caso idêntico exponencial (\(\R(t)=e^{-\lambda t}\)):
\[
\R_{sys}(t)=1-\bigl(1-e^{-\lambda t}\bigr)^n
\]
\begin{itemize}
  \item Paralelo aumenta confiabilidade (e disponibilidade), \textbf{mas} depende de detecção/failover e da ausência de causa comum.
\end{itemize}
% TALKING POINT: custo/complexidade de redundância; introduzir "voter" e "N-modular redundancy".
\end{frame}

\begin{frame}{k-out-of-n e votação (ex.: TMR)}
Para \(n\) componentes e o sistema operar se pelo menos \(k\) estiverem OK:
\[
\R_{k/n}(t)=\sum_{j=k}^{n}\binom{n}{j}\R(t)^j\bigl(1-\R(t)\bigr)^{n-j}
\]
\begin{itemize}
  \item TMR (Triple Modular Redundancy): \(k=2, n=3\) com \textbf{votador}.
  \item Atenção: o \textbf{votador} vira componente crítico (pode dominar série).
\end{itemize}
% TALKING POINT: apontar como isso aparece em hardware crítico (safety) e em sistemas distribuídos (quorum).
\end{frame}

\begin{frame}{Disponibilidade com redundância: exemplo numérico}
Dois servidores idênticos, paralelo ativo, cada um com:
\[
\MTBF=500h,\quad \MTTR=1h \Rightarrow \A = \frac{500}{501}=0.9980
\]
Assumindo independência e failover “perfeito” (aprox.):
\[
\A_{par}\approx 1-(1-\A)^2
\]
\begin{itemize}
  \item Compare o ganho com o caso em série (dependência “AND”) onde a disponibilidade cai.
\end{itemize}
% TALKING POINT: discutir por que "failover perfeito" raramente é real (detecção, split brain, dados).
\end{frame}

\begin{frame}{Common-cause failures (CCF): quando redundância não salva}
\begin{itemize}
  \item Um modelo simples: \textbf{$\beta$-factor}
    \begin{itemize}
      \item fração \(\beta\) das falhas é comum a todos os canais (derruba todos ao mesmo tempo);
      \item fração \(1-\beta\) é independente.
    \end{itemize}
  \item Implicações:
    \begin{itemize}
      \item redundância sem diversidade pode ter ganho muito menor que o esperado.
      \item práticas: diversidade de versão, zona/região, fornecedor, config, caminho de rede.
    \end{itemize}
\end{itemize}
\footnotesize{Referência didática: ETH Zürich (Dependent Failures).}
% TODO link: https://www.ises.ethz.ch/.../07-Dependent-Failures.pdf
% TALKING POINT: ligar com Security: ataque explorando vulnerabilidade comum -> CCF intencional.
\end{frame}

% =========================================================
\section{Modelos de Estado (Markov/CTMC) para reparo e failover}

\begin{frame}{Quando RBD não basta}
RBD é ótimo para estrutura “estática”, mas falha ao modelar:
\begin{itemize}
  \item \textbf{reparo} (volta do componente ao sistema),
  \item \textbf{standby} (frio/morno/quente),
  \item \textbf{cobertura de detecção} (coverage),
  \item \textbf{tempos não desprezíveis} de failover, reboot, rollback.
\end{itemize}
\begin{block}{Ferramenta padrão}
Cadeias de Markov (CTMC) para disponibilidade/confiabilidade em sistemas reparáveis.
\end{block}
% TALKING POINT: reforçar suposição Markoviana (tempos exponenciais) e como aproximar com fases.
\end{frame}

\begin{frame}{Exemplo CTMC: 1-out-of-2 com reparo}
\small
\begin{columns}[T,onlytextwidth]
\column{0.52\textwidth}
Estados:
\begin{itemize}
  \item \(S_2\): 2 operacionais
  \item \(S_1\): 1 operacional
  \item \(S_0\): 0 operacionais (down)
\end{itemize}
Transições (aprox.):
\begin{itemize}
  \item falhas: \(S_2 \to S_1\) taxa \(2\lambda\), \(S_1 \to S_0\) taxa \(\lambda\)
  \item reparos: \(S_1 \to S_2\) taxa \(\mu\), \(S_0 \to S_1\) taxa \(2\mu\)
\end{itemize}

\column{0.48\textwidth}
\centering
\begin{tikzpicture}[
  node distance=10mm,
  st/.style={draw, circle, minimum size=8mm},
  arr/.style={-Latex, thick}
]
\node[st] (S2) {$S_2$};
\node[st, below=of S2] (S1) {$S_1$};
\node[st, below=of S1] (S0) {$S_0$};

\draw[arr] (S2) -- node[right] {$2\lambda$} (S1);
\draw[arr] (S1) -- node[right] {$\lambda$} (S0);

\draw[arr] (S1) -- node[left] {$\mu$} (S2);
\draw[arr] (S0) -- node[left] {$2\mu$} (S1);
\end{tikzpicture}

\end{columns}
\footnotesize{Referência de curso: ETH Zürich (Markov Models).}
% TODO link: https://www.ises.ethz.ch/.../10-Markov-Models.pdf
% TALKING POINT: cobertura imperfeita: adicionar transição S2->S0 com probabilidade (1-c) no failover.
\end{frame}

\begin{frame}{Como extrair disponibilidade estacionária do CTMC}
\begin{itemize}
  \item Resolva \(\pi Q = 0\) com \(\sum \pi_i=1\) (distribuição estacionária).
  \item Disponibilidade: \(\A_\infty = 1 - \pi_{S_0}\).
\end{itemize}
\begin{block}{Mensagem prática}
Modelos de estado permitem incorporar:
\begin{itemize}
  \item \textbf{tempo de failover}, \textbf{coverage}, \textbf{reparo limitado} (fila), \textbf{standby}.
\end{itemize}
\end{block}
% TALKING POINT: se a turma já viu álgebra linear, dá para resolver um caso 3x3 em aula.
\end{frame}

% =========================================================
\section{Software: práticas de engenharia para elevar dependabilidade}

\begin{frame}{Meios para atingir dependabilidade (classificação clássica)}
\begin{enumerate}
  \item \textbf{Fault prevention}: evitar introduzir faults (processo, métodos formais, boas práticas).
  \item \textbf{Fault removal}: encontrar e remover faults (testes, revisão, análise estática).
  \item \textbf{Fault tolerance}: manter serviço correto apesar de faults (redundância, recuperação).
  \item \textbf{Fault forecasting}: estimar incidência/consequências (modelagem, testes, dados de campo).
\end{enumerate}
\footnotesize{Fonte: Avizienis et al. (2004) e materiais de cursos europeus de dependability.}
% TODO link Avizienis: https://www.landwehr.org/2004-aviz-laprie-randell.pdf
% TODO link Pisa: https://docenti.ing.unipi.it/.../BasicConceptsP3.pdf
\end{frame}

\begin{frame}{Observabilidade e MTTR: o “ciclo” operacional}
\begin{itemize}
  \item \textbf{Detectar} rápido (reduz MTTD): métricas, logs estruturados, traces, alertas.
  \item \textbf{Diagnosticar} (reduz MTTI): correlação, runbooks, grafos de dependência.
  \item \textbf{Mitigar} (reduz MTTM): rollback automático, feature flags, auto-healing.
\end{itemize}
\begin{block}{Impacto direto}
Mesmo com \(\MTBF\) igual, reduzir \(\MTTR\) melhora \(\A\) de forma significativa.
\end{block}
% TALKING POINT: relacionar com engenharia de confiabilidade (RE) e SRE; incident response e postmortems.
\end{frame}

\begin{frame}{SLO/Error Budget: “engenharia orientada a metas”}
\begin{itemize}
  \item Defina \(\SLI\): proporção de eventos bons / total.
  \item Defina \(\SLO\) (meta interna) e \(\SLA\) (contratual).
  \item \textbf{Error budget} \(= 1-\SLO\): “quanto você pode errar” na janela.
\end{itemize}
\begin{block}{Regra importante}
Você é tão disponível quanto a soma do risco das suas dependências.
\end{block}
\footnotesize{Fonte: Google SRE (The Calculus of Service Availability; SRE Book/Workbook).}
% TODO links:
% https://sre.google/static/pdf/calculus_of.pdf
% https://sre.google/workbook/alerting-on-slos/
% TALKING POINT: discutir política de congelar mudanças quando estoura orçamento (exceto security fixes).
\end{frame}

\begin{frame}{Tolerância a falhas: padrões de software (com armadilhas)}
\begin{itemize}
  \item \textbf{Timeouts} e \textbf{retries} (com jitter/backoff) para falhas transientes.
  \item \textbf{Circuit breaker} para evitar falha em cascata.
  \item \textbf{Bulkheads} (isolamento) para conter propagação.
  \item \textbf{Graceful degradation}: modos de serviço parcial.
\end{itemize}
% TALKING POINT: retry pode amplificar outage (thundering herd) — precisa rate limit e backoff.
% TODO: inserir figura simples de cascata (link livre) ou desenhar com TikZ.
\end{frame}

% =========================================================
\section{Modelos de confiabilidade em software (crescimento e limites)}

\begin{frame}{Software Reliability Growth: por que difere de hardware?}
\begin{itemize}
  \item Em hardware, desgaste físico justifica hazard crescente (Weibull \(\beta>1\)).
  \item Em software, falhas tendem a vir de \textbf{faults latentes} ativados por certos perfis de uso.
  \item Em testes/campo, correções removem faults \(\Rightarrow\) taxa de falha pode \textbf{decrescer} ao longo do tempo.
\end{itemize}
% TALKING POINT: “software não desgasta”, mas pode piorar com mudanças, dependências e configuração.
\end{frame}

\begin{frame}{Exemplo clássico: NHPP (Goel--Okumoto)}
\small
Modelo assume que o número cumulativo de falhas segue um processo de Poisson não homogêneo:
\[
\mu(t)=N\left(1-e^{-bt}\right)
\]
onde \(N\) é o número esperado total de falhas “eventuais” e \(b\) controla a rapidez do decaimento.
\begin{itemize}
  \item Bom para dados de teste onde falhas vão diminuindo com correções.
  \item Precisa cuidado: pressupõe condições relativamente estáveis e coleta consistente.
\end{itemize}
\footnotesize{Fonte: notas de aula universitárias (Software Reliability Estimation, Univ. of Denver).}
% TODO link: https://www.cs.du.edu/~snarayan/sada/teaching/COMP3705/lecture/p1/W63.pdf
% TALKING POINT: discutir quais dados são necessários (tempos entre falhas / contagens por intervalo).
\end{frame}

\begin{frame}{Musa--Okumoto (logarítmico) e leitura crítica}
\small
Intuição: consertos tardios produzem menor ganho; intensidade decresce com falhas acumuladas.
\begin{itemize}
  \item Modelos SRGMs são úteis para \textbf{planejamento} e \textbf{comparação}, não como “verdade absoluta”.
  \item Sempre verifique:
    \begin{itemize}
      \item viés de coleta (mudança de carga/perfil),
      \item mudanças arquiteturais,
      \item qualidade do processo de correção (debugging imperfeito),
      \item dependências externas (que mudam).
    \end{itemize}
\end{itemize}
\footnotesize{Fonte: Univ. of Denver (W63) e literatura clássica de SRGM.}
% TODO link: https://www.cs.du.edu/~snarayan/sada/teaching/COMP3705/lecture/p1/W63.pdf
\end{frame}

% =========================================================
\section{Exemplos aplicados (mais realistas)}

\begin{frame}{Exemplo 1: serviço embarcado de telemetria (com MTTR decomposto)}
Dados:
\[
\MTBF=500h,\quad \mathrm{MTTD}=10min,\quad \mathrm{MTTI}=20min,\quad \mathrm{MTTM}=30min
\]
\[
\MTTR\approx 60min = 1h \Rightarrow \A=\frac{500}{501}=99.80\%
\]
\begin{itemize}
  \item Se automatizar detecção e rollback: \(\mathrm{MTTD}=1min\), \(\mathrm{MTTM}=5min\) \(\Rightarrow\) \(\MTTR\downarrow\) forte.
\end{itemize}
% TALKING POINT: o que reduz cada parcela (observabilidade vs automação vs processos).
\end{frame}

\begin{frame}{Exemplo 2: watchdog + estado seguro + perda de serviço}
\begin{itemize}
  \item Fault de firmware leva a deadlock.
  \item Watchdog reinicia em 5s, mas:
    \begin{itemize}
      \item reinit de periféricos (mais 10s),
      \item ressincronização de estado (mais 15s).
    \end{itemize}
  \item Downtime efetivo por evento: 30s.
\end{itemize}
\begin{block}{Pergunta de engenharia}
O que é “falha” para o usuário? Reiniciar em 5s pode ser ótimo, mas perder estado pode ser inaceitável em alguns domínios.
\end{block}
% TALKING POINT: preparar gancho para Safety/Resiliência: fail-safe, fail-operational, degraded modes.
\end{frame}

\begin{frame}{Exemplo 3: SLO 99.9\% e dependências (orçamento por componente)}
\begin{itemize}
  \item SLO do serviço A: 99.9\% \(\Rightarrow\) budget = 0.1\% por janela.
  \item Se A tem \(N\) dependências críticas, o budget precisa ser \textbf{particionado} (heurística comum).
  \item Resultado prático: dependências precisam ter “mais 9s” de disponibilidade que o serviço final.
\end{itemize}
\footnotesize{Fonte: Google SRE (The Calculus of Service Availability).}
% TODO link: https://sre.google/static/pdf/calculus_of.pdf
% TALKING POINT: comparar com aproximação em série (produto) e mostrar por que dependências dominam.
\end{frame}

% =========================================================
\section{Exercícios (nível engenharia)}

\begin{frame}{Exercícios 1: confiabilidade e hazard}
\begin{enumerate}
  \item Para \(\lambda = 2\times10^{-4}\,h^{-1}\), calcule \(\R(100h)\) e \(\MTTF\).
  \item Uma Weibull com \(\beta=2\), \(\eta=1000h\): calcule \(\R(500h)\) e interprete o hazard.
  \item Mostre (algebricamente) que hazard constante implica \(\R(t)=e^{-\lambda t}\).
\end{enumerate}
% TALKING POINT: em sala, pedir que identifiquem em que tipo de componente \beta>1 é esperado.
\end{frame}

\begin{frame}{Exercícios 2: disponibilidade e trade-offs}
\begin{enumerate}
  \item Um sistema tem \(\MTBF=1200h\) e \(\MTTR=4h\). Qual \(\A\)?
  \item Se você puder escolher entre:
    \begin{itemize}
      \item reduzir \(\lambda\) em 10\% (melhorando robustez), ou
      \item reduzir \(\MTTR\) em 50\% (melhorando observabilidade/rollback),
    \end{itemize}
    qual ação dá maior ganho em \(\A\) neste caso?
  \item Calcule o downtime mensal permitido para 99.95\% e compare com 99.9\%.
\end{enumerate}
% TALKING POINT: reforçar que MTTR é muitas vezes mais “barato” de reduzir do que MTBF.
\end{frame}

\begin{frame}{Exercícios 3: composição e dependências}
\begin{enumerate}
  \item Dois componentes em série com \(\A_1=0.999\) e \(\A_2=0.9995\). Aproxime \(\A_{sys}\).
  \item Dois componentes em paralelo ativo (independentes) com \(\A=0.999\). Aproxime \(\A_{par}\).
  \item Com causa comum \(\beta=0.05\), discuta qualitativamente por que o ganho do paralelo cai.
\end{enumerate}
% TALKING POINT: pedir uma discussão de engenharia: "o que você faria para reduzir beta?"
\end{frame}

% =========================================================
\section{Resumo e gancho para Safety, Security e Resiliência}

\begin{frame}{Resumo}
\begin{itemize}
  \item \textbf{Confiabilidade} modela “não falhar no intervalo”; \textbf{disponibilidade} inclui \textbf{reparo} e \textbf{recuperação}.
  \item Exponencial é um \textbf{caso} (hazard constante); Weibull e hazard ajudam a capturar realidade.
  \item Redundância melhora muito, mas \textbf{dependências} e \textbf{common-cause} podem dominar.
  \item Em sistemas modernos, \textbf{MTTR} é alavanca enorme via observabilidade e automação.
  \item Metas SLO/error budget conectam modelagem com operação contínua e trade-offs de mudança.
\end{itemize}
\end{frame}

\begin{frame}{Gancho: próxima aula (Safety, Security e Resiliência)}
\begin{itemize}
  \item A taxonomia de dependabilidade inclui \textbf{Safety} e \textbf{Security} como atributos/ameaças complementares.
  \item \textbf{Safety}: falhas podem causar dano físico \(\Rightarrow\) foco em hazard, risco, estados seguros, certificação.
  \item \textbf{Security}: faults podem ser \textbf{maliciosos} (ataques) \(\Rightarrow\) integridade/confidencialidade + disponibilidade sob ataque.
  \item \textbf{Resiliência}: capacidade de \textbf{manter e recuperar} serviço sob falhas/ataques/degradação e dependências.
\end{itemize}
% TALKING POINT: antecipar que "resiliência" mistura engenharia + operação + resposta a incidentes (e também security fixes).
\end{frame}

\begin{frame}{Referências essenciais (para leitura e credibilidade)}
\small
\begin{itemize}
  \item Avizienis, Laprie, Randell, Landwehr — \textit{Basic Concepts and Taxonomy of Dependable and Secure Computing} (IEEE TDSC, 2004).\\
  \footnotesize{\url{https://www.landwehr.org/2004-aviz-laprie-randell.pdf}}
  \item ETH Zürich — \textit{Reliability of Technical Systems} (distributions, dependent failures, Markov).\\
  \footnotesize{\url{https://www.ises.ethz.ch/education/lectures/reliability-of-technical-systems.html}}
  \item NIST — \textit{Reliability/Availability Handbook} (exponencial/HPP, MTBF e intervalos de confiança).\\
  \footnotesize{\url{https://www.itl.nist.gov/div898/handbook/apr/}}
  \item MIT (notes/OCW) — materiais de confiabilidade/bathtub e fundamentos.\\
  \footnotesize{\url{https://web.mit.edu/2.611/www/notes/BB.pdf}}
  \item University of Pisa — slides de Dependability (cadeia fault-error-failure, métodos).\\
  \footnotesize{\url{https://docenti.ing.unipi.it/~a008669/}}
  \item Google SRE — \textit{The Calculus of Service Availability} e Workbook (SLO/alerting).\\
  \footnotesize{\url{https://sre.google/static/pdf/calculus_of.pdf}}
\end{itemize}
% TODO: se você quiser, posso adaptar essa lista para ABNT/Vancouver e gerar um slide "Bibliografia" formal.
\end{frame}

\end{document}